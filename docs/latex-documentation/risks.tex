\section{Risks and techincal debt}
Potential problems, known technical risks, technical debt

\subsection{Database performance}
Storage capacity

\subsection{Riot API Ratelimits}

\subsection{Changes in Riot API}

\subsection{Sicherheit}

Derzeit ist eine Authentifizierung implementiert, die mit Hilfe von JWT-Zugriffstokens bestimmte Seiten vor unbefugten Zugriff schützen.
Jedoch ist dies bei anderen API-Endpunkten ('Player-Backend') nicht der Fall. Ebenso existiert keine Zugriffskontrolle zwischen den
einzelnen Microservice-Diensten, um zu verifizieren, dass der Anfragende befugt ist. Hierfür können Zertifikate, die jeweils ein
Dienst besitzt, die dienstübergreifende Kommunikation mehr Sicherheit bieten. Ebenso wird dadurch die Kommunikation untereinander
verschlüsselt.

Ebenso fehlt ein Ratenlimit im Frontend beziehungsweise im Backend der einzelnen Dienste. Dadurch kann ein Denial-of-Service-Angriffe (DoS)
nicht verhindert werden oder ungewollte Auslastung der Netzwerkbandbreite. Dies ist vor allem bei der Schnittstelle kritisch, die für das 
Importieren von Spieler-Daten verantwortlich ist. Aufgrund der rechenintensiven Berechnung und Importierung der Spielerdaten kann 
eine erhöhte Auslastung zur Unstabilität des Dienstes führen.