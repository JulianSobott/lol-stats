\section{Architekturentscheidungen}

\subsection{Service orientierte Architektur}\label{subsec:soa}

Das Backend soll in einzelne Services aufgeteilt werden.

\subsubsection{Gründe für die Entscheidung}\label{subsubsec:grunde-fur-die-entscheidung}

\textbf{Schnelle Entwicklung}: Dadurch, dass jeder Entwickler an einem Service arbeitet, gibt es kaum Abhängigkeiten
und weniger Merge Konflikte.
Da es keine Microservice Architektur ist, gibt es zwischen den Services kaum Kommunikation, welches das Entwickeln vereinfacht.

\textbf{Freie Technologiewahl}: Für jeden Service kann die Technologie gewählt werden, die am besten geeignet ist.

\textbf{Schnelles und unabhängiges Deployment}: Da die Services kaum Abhängigkeiten haben, können die Services
schnell und einfach deployt werden.

\subsubsection{Mögliche Alternativen}\label{subsubsec:mogliche-alternativen}

\textbf{Monolith}: Es wurde sich gegen ein Monolith entschieden, um eine unabhängige Entwicklung zu gewährleisten.

\textbf{Microservices}: Es wurde sich gegen eine strenge Microservices Architektur entschieden, weil diese schwieriger
zu designen ist und komplexer in der Kommunikation ist.


\subsection{grpc für die Kommunikation zwischen Player API und Games Importer}\label{subsec:grpc-fur-die-kommunikation-zwischen-player-api-und-games-importer}

Für die Kommunikation zwischen Player API und Games Importer soll grpc verwendet werden.

\subsubsection{Gründe für die Entscheidung}\label{subsubsec:grunde-fur-die-entscheidung-01}

\textbf{Streaming}: Durch grpc ist es sehr einfach, Streaming zu verwenden.
Streaming ist hier sinnvoll, da es eine Verbindung gibt, über die immer wieder Updates übertragen werden sollen.

\textbf{Einfache Implementierung}: Mithilfe der Client-Bibliotheken ist die Implementierung sehr einfach.
Man muss in einer .proto Datei nur das Schema festlegen und daraus wird automatisch der entsprechende Code generiert.
Im eigenen Code muss man dann nur noch die generierte Methode aufrufen und die restliche Kommunikation von der Bibliothek übernommen.

\subsubsection{Mögliche Alternativen}\label{subsubsec:mogliche-alternativen-01}

\textbf{Sockets}: Ist sehr low level und man muss sich um mehr kümmern.

\textbf{Message Queue}: Hat nur für die Verwendung einer Methode einen zu großen Overhead.

\textbf{Polling}: Ist nicht so effizient, da es entweder mehr Ressourcen braucht oder nicht real-time ist.