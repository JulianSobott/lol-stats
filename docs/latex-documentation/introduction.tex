\section{Einführungen und Ziele}
\subsection{Ziel}

Lol-Stats ist eine Webanwendung, die speziell für aktive "League of Legends"-Spieler entwickelt wurde. Ihre Hauptfunktionalität liegt in der Berechnung und Vermittlung
von Statistiken, die innerhalb der Spielrunden erfasst werden. Bei den Statistiken handelt sich nicht nur um übliche, simple Daten, sondern vielmehr um spezielle Errungenschaften, die es so nicht auf anderen Webseiten gibt. 
Lol-Stats bietet außerdem die Möglichkeit, diese Errungenschaften mit denen anderer Spieler zu vergleichen. Man kann bestimmte Spieler auch als Konkurrenten hinzufügen, um sich schneller und einfacher mit ihnen zu vergleichen.\\
Zudem kann man sich mit der gesamten Gruppe an ausgewählten Konkurrenten oder der globalen Liste an Spieler vergleichen.

\subsection{Anforderungen}

\subsubsection{Technisch}

Die gesamte Architektur soll dabei geschickt in Services aufgeteilt werden, wobei jeder Service in seinem eigenen Docker-Container laufen soll.
Ebenso gibt es kein zentrales Datenbanksystem, denn jeder Service soll, wenn sinnvoll, über eine eigene Datenbank verfügen.\\
Das Backend ist dafür zuständig die Daten aus der Riot API zu importieren und dem Frontend zur Verfügung zu stellen. Dazu muss eine Schnittstelle zwischen der Riot API und dem Backend, sowie zwischen Backend und Frontend geschaffen werden. Des weiteren existiert ein Servive zur Verwaltung der User-Accounts, welcher ebenfalls mit dem Frontend kommunizieren muss.\\
In dieser Anwendung soll es unter anderem Echtzeitverarbeitung von Daten geben. 
Das Frontend soll eine Benutzerschnittstelle implementieren, die Responsiveness und Intuitivität aufweist, sodass der Umgang mit der Website für Benutzer möglichst einfach ist.

\subsubsection{Fachlich}

\begin{itemize}
    \item Als Nutzer will ich die Achievements von meinem gewünschten Account sehen können.
    \item Als Nutzer will ich meine Achievements mit den Achievements beliebiger Spieler, global oder mit Ranks vergleichen können.
    \item Als Nutzer will ich beliebige Spieler zu meinen Favoriten hinzufügen oder löschen können.
    \item Als Nutzer will ich mich speziell nur mit diesen Favoriten vergleichen können.
    \item Als Nutzer will ich meine zuletzt gespielten Spiele ansehen können.
    \item Als Nutzer will ich globale, zufällige Statistiken sehen können, auch wenn ich nicht angmeldet bin.
\end{itemize}

