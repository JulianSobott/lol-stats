\section{Building block view}
UML package and class diagrams (perhaps auto-generate with tool, but ensure clean and well-organized)
Include description of responsibility and name of each package and class.
List external software used and provide links


\subsection{Games Importer Service}
\subsubsection{Class Diagrams}
\graphic{GamesImporter}{Klassendiagramm des Game Importer Service}
\subsubsection{Responsibilities}
Dieser Service ist dafür zuständig, die benötigten Daten bei der Riot API anzufragen und dann in der LoL Datenbank zu speichern. Die Struktur des Services ist im obigen Klassendiagramm zu sehen. \\
Die db Klasse bietet alle Funktionen, um mit der Datenbank zu interagieren: neue Daten speichern, Daten lesen und Daten updaten. Dazu wird das psycopg2 Package verwendet.\\
Die App Klasse beinhaltet den main loop (update\_loop Funktion). In dieser werden dann weitere Funktionen aufgerufen, um alle Spieler zu updaten. Dazu werden zuerst die Methoden des summoner.py Moduls verwendet, welche einen einzelnen Spieler updaten können. In dieser Methode werden zunächst alle Daten des Summoners (Level, Name, ...) aktualisiert. Dann wird das matchhistory.py Modul verwendet, welches für den Spieler die MatchHistory importieren kann (add\_missing\_games\_tp\_db(db, get\_match\_ids(), puuid). Dieses verwendet dabei noch die Challenges Klasse, welche die Challenges für einzelne Spieler speichern kann (store\_challenges()). Dabei werden mithilfe der in der Datenbank vorhandenen Werte für total, average und highscore die neuen Werte berechnet und diese dann gespeichert.\\

Der PlayerImportRequest erbt von der von GRPC generierten Klasse ImporterService und ist für die Verbindung mit der PlayerAPI zuständig. Über diese Klasse kann eine Anfrage gesendet werden, um den Import eines neuen Spielers zu starten, wobei dann dieselbe Abfolge durchlaufen wird wie oben bereits beim updaten eines Spielers beschrieben.

\subsubsection{External Software}
GRPC: \href{https://grpc.io/docs/languages/python/basics/}\\
Cassiopeia: \href{https://github.com/meraki-analytics/cassiopeia}\\
Riot Watcher: \href{https://github.com/pseudonym117/Riot-Watcher}\\
psycopg2: \href{https://pypi.org/project/psycopg2/}\\
Riot API: \href{https://developer.riotgames.com/}\\

\subsection{Player API}
\subsubsection{Class Diagrams}
\subsubsection{Responsibilities}
\subsubsection{External Software}
\subsection{User Management}
\subsubsection{Class Diagrams}
\subsubsection{Responsibilities}
\subsubsection{External Software}

\newpage

\subsection{Frontend}

\subsubsection{Mockups}

Beim Planungsprozess wurden folgende Mockups erstellt.

\graphic{mockups/mockup_all.png}{lle Mockups des Frontends}

\subsubsection{Class Diagrams}

\graphic{Frontend_UML.png}{Das UML-Diagramm des Frontends}

\subsubsection{Responsibilities}

Die Benutzeroberfläche (Frontend) dient als Client und ist für die Kommunikation mit dem verschiedenen Microservices zuständig. Es stellt die Daten übersichtlich dar und sorgt für eine
angenehmes Nutzererlebnis. \\

\textbf{sign-up.vue}: Diese Seite ist für die Registrierung von neuen Benutzer:innen verantwortlich. Sobald die eine E-Mail und Passwörter übergeben wurden, so werden diese zunächst
validiert. Dies schließt die Überprüfung einer korrekten E-Mail-Adresse sowie die ausrechende Länge des Passwortes mit ein. Die Methode \verb|register()| sendet eine HTTP-Anfrage mit den 
entsprechenden Daten an das User-Backend, wo das Benutzerkonto erstellt wird.
\newline

\textbf{login.vue}: Zum Anmelden eines Benutzers muss auf dieser Seite die korrekten Anmeldedaten in das Formular eingegeben werden. Die Methode \verb|login()| prüft, ob das User-Backend
verifiziert hat, dass die Anmeldedaten korrekt sind. Ist dies der Fall, so wird der Token, das das User-Backend zusendet, gespeichert. 
\newline

\textbf{setting.vue}: Bei der ersten Anmeldung müssen die Nutzer:innen den gewünschten Spielernamen übergeben. Durch das Speichern dieser Einstellungen durch die Methode \verb|savePlayername|
sendet das Frontend den Spielernamen an das User-Backend. Anschließend werden die Nutzer:innen an die Hauptseite weitergeleitet. Im Zuge der Speicherung wird gleichzeitig ein "Import"-Befehl
an das "Player-Backend" gesendet, sodass die Spielerinformationen direkt importiert werden und zur Verfügung stehen. Die Methode \verb|deleteAccount| ist für das Löschen eines Kontos zuständig.
\newline

\textbf{index.vue}: Die Hauptseite stellt mithilfe der Methode \verb|fetchRandomStats| Leaderboard-Statistiken dar. Hierfür das "Player-Backend" angefragt, welches zufällige 
Leaderboards zurückgibt.
\newline

\textbf{dashboard.vue}: Das Dashboard stellt alle Informationen zu der jeweiligen LOL-Spieler:in dar. Die Methode \verb|fetchUserData| ruft die Spielerdaten anhand des gespeicherten Tokens
von dem "Player-Backend" ab. Ebenso wird geprüft, ob die Account bereits importiert wird. Ist das nicht der Fall, so wird ein Ladebalken dargestellt, der den Import-Fortschritt aufzeigt. Dies
wird durch eine regelmäßige Anfrage an das "Player-Backend" ermöglicht, welches die Aktuellen importieren Spiele sowie den Fortschritt in Prozent zurückgibt.
\newline

\textbf{competitors.vue}: Auf der "Competitors"-Seite können Spieler:innen zum Vergleichen von Statistiken hinzugefügt oder gelöscht werden. Die Methode \verb|getCompetitors| ruft die gespeicherten
Freunde des Spielers ab und listet sie auf. Die Methode \verb|removeCompetitor| entfernt den jeweiligen Gegner aus der Freundesliste.
\newline

\textbf{achievements.vue}: Auf dieser Seite kann der eigene Account mit anderen Personen oder Ranks verglichen werden. Die einzelnen "Challanges" werden in Tabellen aufgelistet und in Tabs unterteilt.
Das Aufrufen der Seite führt die Methode \verb|fetchAchievements| aus, die eine Anfrage an das "Player-Backend" sendet. Mithilfe eines Filtersystems können die persönlichen
Statistiken zwischen den Freundeslisten, Global oder nur mit einer gewünschten Person verglichen werden. Falls der zu vergleichende Spieler nicht importiert wurde, kann auch von dort dieser über einen 
Button importiert werden, wodurch die Methode \verb|importPlayer| aufgerufen wird. Außerdem ist es möglich, bestimmte Kategorien als Favorit abzuspeichern. Diese übernimmt die Methode
\verb|toggleFavoriteAchivement|, die beim Betätigen der Favoriten-Buttons ausgelöst wird. Die Methode \verb|filterApplied| sendet eine aufbereitete HTTP-Anfrage mit den entsprechenden Filtern
an das "Player-Backend", sobald die gewünschten Filter übernommen wurden.
\newline

\subsubsection{External Software}