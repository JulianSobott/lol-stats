\section{Crosscutting concepts}
Persistence 
E-R Diagram, information architecture model, etc.
REST APIs
List the backend REST APIs. For more detail you can refer to generated API doc.
User Interface
E.g.., SPA, responsive, etc.
Communication/Integration with other Software Systems (external APIs)
Configurability

\subsection{LoL Database ER-Diagram}
\graphic{LoLDatabase}{ER-Diagramm der LoL Datenbank}
In dieser Datenbank werden alle League of Legends bezogenen Daten gespeichert. Das sind alle Daten die dann auch im Frontend angezeigt werden. Die Daten werden vom Games Importer Service in die Datenbank geschrieben und dann von der Player API für das Frontend zur Verfügung gestellt.\\
Tabellen in der Datenbank:\\
\begin{itemize}
\item \textbf{Summoners:} Daten für einen Spieler
\item \textbf{Games:} Daten für einzelne Spiele
\item \textbf{Challenges:} Werte des Spieler in einer Challenge
\item \textbf{ChallengeClasses:} Verschiedene Challenges, die in der Challenges Tabelle auftauchen können, mit genauerer Beschreibung und Einteilung in Klassen
\item \textbf{SummonerSpells:} Klasse für Mapping von SummonerSpell IDs zu Name und Icon URL
\item \textbf{Champions:} Klasse für Mapping von Champion IDs zu Name und Icon URL
\item \textbf{Items:} Klasse für Mapping von Item IDs zu Name und Icon URL
\item \textbf{SummonerIcons:} Klasse für Mapping von Icon IDs zu URL
\item \textbf{Patches:} Hier wird gespeichert, welche Patches von League of Legends bereits in der Datenbank eingetragen wurden. Da mit jedem Patch z.B. neue Icons dazukommen, müssen die Tabellen nach jedem Patch geupdated werden. Nach dem update wird in dieser Tabelle festgehalten, wann die Datenbank für welchen Patch geupdated wurde.
\end{itemize}

\subsection{User Database ER-Diagram}

\subsection{User Management API Specification}

\subsection{Player API Specification}s

\subsection{GRPC Interface}
Damit der Import eines neuen Spieler angefragt werden kann, muss eine Kommunikation zwischen Frontend und dem Games Importer Service bestehen. Das Frontend kommuniziert nur mit der PlayerAPI, welche die Anfrage dann an den Games Importer Service weiterleiten muss.\\
Da der Player API Service und der Games Importer Service beide in Pyton geschrieben wurden, kann diese Kommunikation über GRPC erfolgen, welches es der PlayerAPI erlaubt über einen Protocol Request die Import Funktion des Game Importers aufzurufen.\\
Die Schnittstelle wird in einer .proto Datei definiert:

\begin{lstlisting}
syntax = "proto3";

service Importer {
  rpc import_player (ImportRequest) returns (stream ImportReply) {}
}

message ImportRequest {
  string puuid = 1;
}

message ImportReply {
  int32 games_imported = 1;
  int32 total_games = 2;
}
\end{lstlisting}
Der Request beinhaltet hierbei die ID des zu importierenden Spielers und die Reply den aktuellen Fortschritt des Vorgangs. Die Reply wird solange gesendet, bis der Spieler vollständig importiert wurde.\\
Mithilfe dieser Datei werden dann die nötigen Python Klassen generiert. In der \textit{PlayerImportRequest.py} wird dann die \textit{import\_player(request, context)} Methode implementiert, welche dann beim erhalten eines Requests aufgerufen wird und im request Parameter die ID des Spieler erhält. Dann kann der Games Importer diesen importieren und mittels eines \textit{yield} Statements in der \textit{import\_player()} Funktion die Replies an die Player API senden.

\subsection{Configurability and Traffic}